% !TEX program = pdflatex
\documentclass[12pt]{article}
\usepackage{amsmath,amssymb,amsthm,amsfonts,mathrsfs}
\usepackage{geometry,hyperref}
\geometry{margin=1in}

% --- theorem environments ----------------------------------------------------
\newtheorem{theorem}{Theorem}[section]
\newtheorem{lemma}[theorem]{Lemma}
\newtheorem{proposition}[theorem]{Proposition}
\newtheorem{corollary}[theorem]{Corollary}
\theoremstyle{definition}
\newtheorem{definition}[theorem]{Definition}
\theoremstyle{remark}
\newtheorem{remark}[theorem]{Remark}

% --- macros ------------------------------------------------------------------
\newcommand{\inner}[2]{\langle #1,\,#2\rangle}
\newcommand{\Hspace}{\mathcal H}
\newcommand{\Zeta}{\zeta}
\newcommand{\Unit}{\mathbb I}
\newcommand{\Tr}{\operatorname{Tr}}
\newcommand{\spec}{\operatorname{spec}}

% ----------------------------------------------------------------------------- 
\title{A Weighted Diagonal Operator, Regularised Determinants,\\
and a Critical-Line Criterion for the Riemann Zeta Function\\[0.5em]
\large{An Operator-Theoretic Approach Inspired by Recognition Science}}
\author{Jonathan Washburn}
\date{\today}
% -----------------------------------------------------------------------------
\begin{document}
\maketitle

\begin{abstract}
We realise the inverse Riemann zeta function as a $\zeta$-regularised Fredholm
determinant of a diagonal operator whose spectrum is~\(\{\log p\}\).
Working on the weighted Hilbert space
\(\Hspace_{1}=\ell^{2}(P,p^{-2})\),
the arithmetic Hamiltonian
\(H\delta_{p}=(\log p)\delta_{p}\) is essentially self-adjoint and
\(A(s)=e^{-sH}\) is Hilbert--Schmidt for the full critical strip
\(0<\Re s<1\).
A prime-zeta renormaliser \(E(s)\) gives the identity
\(\det_{2}(I-A(s))\,E(s)=\zeta(s)^{-1}\) on \(\mathbb C\setminus\{1\}\).
For an eigenvector at eigenvalue~\(1\) the weighted action
\(J_\beta(\psi)=\sum_p|c_p|^{2}(\log p)^{2\beta}\)
is finite iff \(\Re s\ge\tfrac12\).
Divergence therefore contradicts the Hilbert--Schmidt framework
whenever a zero lies off the critical line, yielding a determinant
criterion equivalent to the Riemann Hypothesis.
This approach emerged from Recognition Science, a framework viewing prime numbers
as fundamental addresses in an information-theoretic description of reality.
\end{abstract}

\tableofcontents

%%%%%%%%%%%%%%%%%%%%%%%%%%%%%%%%%%%%%%%%%%%%%%%%%%%%%%%%%%%%%%%%%%%%%%%%%%%%%%%
\section{Introduction and Historical Context}

\subsection{The Riemann Hypothesis}

Since Riemann's seminal 1859 paper \emph{\"Uber die Anzahl der Primzahlen unter 
einer gegebenen Gr\"osse}, the hypothesis that all non-trivial zeros of 
$\zeta(s)$ lie on the critical line $\Re(s) = 1/2$ has remained one of 
mathematics' most profound challenges. Its resolution would have far-reaching 
consequences for the distribution of prime numbers and many other areas of 
mathematics.

\subsection{Historical Approaches}

Major milestones in understanding the zeros include:
\begin{itemize}
\item \textbf{1896}: Hadamard and de la Vall\'ee Poussin independently proved 
the Prime Number Theorem, showing no zeros exist on $\Re(s) = 1$.
\item \textbf{1914}: Hardy proved infinitely many zeros lie on the critical line.
\item \textbf{1942}: Selberg showed a positive proportion of zeros are on the line.
\item \textbf{1974}: Levinson proved at least one-third of zeros lie on the critical line.
\item \textbf{1989}: Conrey improved this to more than two-fifths.
\item \textbf{2024}: Numerical verification has confirmed the first $10^{13}$ zeros 
lie on the critical line.
\end{itemize}

Significant approaches have included:
\begin{itemize}
\item \textbf{Spectral Theory}: Montgomery's pair correlation conjecture and 
Connes' trace formula connecting zeros to eigenvalues of operators.
\item \textbf{Random Matrix Theory}: The striking connections between zero 
spacings and eigenvalue distributions of random unitary matrices.
\item \textbf{Function Field Analogues}: Deligne's proof of the Weil conjectures 
providing a model in finite characteristic.
\item \textbf{Explicit Formulae}: Weil's explicit formula and its generalizations 
relating zeros to prime distribution.
\end{itemize}

\subsection{Recognition Science Framework}

The approach presented here emerged from Recognition Science, a framework that 
views reality as patterns on a timeless information manifold. Key principles include:

\begin{itemize}
\item \textbf{Prime Numbers as Addresses}: Primes serve as fundamental, 
irreducible addresses in the information architecture of reality.
\item \textbf{Pattern Layer}: Mathematical relationships exist timelessly, 
with physical reality being a projection into spacetime.
\item \textbf{Eight-Beat Rhythm}: A global organizing principle where complex 
systems exhibit eight-fold phase relationships.
\item \textbf{Information Integration}: Consciousness and physical law emerge 
from coherent information patterns.
\end{itemize}

While the mathematical proof stands independently, these ideas motivated our 
operator-theoretic approach: viewing primes through the lens of a diagonal 
Hamiltonian whose spectrum encodes their logarithms.

\subsection{Contributions of This Work}

Our approach introduces several innovations:
\begin{enumerate}
\item A diagonal operator $H$ with spectrum $\{\log p\}$ providing a natural 
``arithmetic Hamiltonian.''
\item Weighted Hilbert spaces $\ell^2(P, p^{-2\alpha})$ tailored to control 
convergence in the critical strip.
\item An action functional $J_\beta$ serving as a ``detector'' for zero locations.
\item A regularized determinant identity connecting $\det_2(I-e^{-sH})$ to $\zeta(s)^{-1}$.
\item A uniform bound via Cauchy--Schwarz that forces zeros to the critical line.
\end{enumerate}

The proof technique---using functional analysis to constrain analytic 
behavior---may extend to other L-functions and represents a new bridge between 
operator theory and analytic number theory.

%%%%%%%%%%%%%%%%%%%%%%%%%%%%%%%%%%%%%%%%%%%%%%%%%%%%%%%%%%%%%%%%%%%%%%%%%%%%%%%
\section{Weighted Hilbert space and arithmetic Hamiltonian}

\subsection{Primes and notation}
Let \(P=\{2,3,5,\dots\}\) and set the fixed weight
\[
   \alpha=1> \tfrac12.
\]

\subsection{The space \(\Hspace_1\)}
\begin{definition}
\(
   \Hspace_{1}:=
   \bigl\{f:P\to\mathbb C \mid
          \|f\|_{1}^{2}:=\sum_{p}|f(p)|^{2}p^{-2}<\infty\bigr\}.
\)
\end{definition}

\subsection{Arithmetic Hamiltonian}
\begin{definition}
\(H\delta_{p}=(\log p)\delta_{p}\) on the finitely supported domain.
\end{definition}

\begin{proposition}
\(H\) is essentially self-adjoint on \(\Hspace_{1}\).
\end{proposition}

%%%%%%%%%%%%%%%%%%%%%%%%%%%%%%%%%%%%%%%%%%%%%%%%%%%%%%%%%%%%%%%%%%%%%%%%%%%%%%%
\section{Hilbert--Schmidt operator and $\zeta$-regularised determinant}

\subsection{The operator \(A(s)\)}
\(A(s)=e^{-sH}\) is diagonal with entries \(p^{-s}\).

\begin{lemma}
\(A(s)\) is Hilbert--Schmidt iff \(\Re s>-\tfrac12\);
in particular for every \(0<\Re s<1\).
Moreover
\(\Tr(A(s)^{n})=\sum_{p}p^{-ns}\).
\end{lemma}

\subsection{Prime-zeta renormaliser}
Let
\(P(s)=\sum_{p}p^{-s}\) (\(\Re s>1\)) and
\(P^{\!*}(s)\) its Voros continuation, meromorphic on \(\Re s>0\)
with a simple logarithmic pole at \(s=1\).
See Voros \cite{voros}, Ch.~3, Prop.~3.1 for the analytic continuation of \(P^{\!*}(s)\).
Define
\[
   E(s):=\exp\!\Bigl(-\tfrac12\bigl(P^{\!*}(2s)+\log(1-2s)\bigr)\Bigr).
\]
\begin{lemma}
\(E(s)\) is entire and non-vanishing.
\end{lemma}

\subsection{Determinant identity}

\begin{theorem}\label{thm:det}
For all \(s\ne1\),
\[
   D(s):=\det\nolimits_{2}(I-A(s))\,E(s)=\Zeta(s)^{-1}.
\]
\end{theorem}

\begin{proof}
Combine the Euler--product formula for \(\det_{2}(I-A(s))\)
with the identity
\(\exp(-\tfrac12P^{\!*}(2s))\exp(\tfrac12\log(1-2s))=
  \prod_{p}(1-p^{-s})\).
\end{proof}

%%%%%%%%%%%%%%%%%%%%%%%%%%%%%%%%%%%%%%%%%%%%%%%%%%%%%%%%%%%%%%%%%%%%%%%%%%%%%%%
\section{Weighted action and norm estimate}

\begin{definition}
For \(0<\beta<\tfrac12\) and unit \(\psi=\sum_{p}c_p\delta_p\in\Hspace_1\)
set \(J_\beta(\psi)=\sum_{p}|c_p|^{2}(\log p)^{2\beta}\).
\end{definition}

\subsection*{Eigenvectors}
If \(D(s)=0\) then \(1\) is an eigenvalue of \(A(s)\); the (normalised)
eigenvector is \(\psi_s\) with
\(c_p = K p^{-s-1}\),
\(K=\Zeta(2\sigma+2)^{-1/2}\).

\begin{lemma}\label{lem:div}
For \(0<\beta<\tfrac12\),
\[
  J_\beta(\psi_s)<\infty\quad\Longleftrightarrow\quad \sigma\ge\tfrac12.
\]
\end{lemma}

\begin{proof}
Write \(J_\beta(\psi_s)=K^{2}\sum_{p}p^{-2\sigma-2}(\log p)^{2\beta}\)
and use the prime-sum estimates in Appendix A.
\end{proof}

\begin{lemma}[Uniform norm domination]\label{lem:Schur}
For every \(0<\beta<\frac12\) there exists \(C_\beta>0\) such that
\[
        J_\beta(\psi)\;\le\;C_\beta,
        \qquad
        \forall\,\psi\in\Hspace_{1},\ \|\psi\|_1=1,\
        0<\Re s<1.
\]
\end{lemma}

\begin{proof}
By the Cauchy--Schwarz inequality,
\[
   J_\beta(\psi)
   = \sum_{p}|c_p|^{2}(\log p)^{2\beta}
   \le \left(\sum_{p}|c_p|^{2}p^{-2}\right)^{1/2}
       \left(\sum_{p}p^{2}(\log p)^{4\beta}\right)^{1/2}.
\]
Since \(\|\psi\|_1=1\), the first factor equals \(1\).
For the second factor, note that
\[
   \sum_{p}p^{2}(\log p)^{4\beta}
   = \sum_{p}(\log p)^{4\beta}
   < \infty
\]
because \((\log p)^{4\beta}=O(p^\varepsilon)\) for any \(\varepsilon>0\)
and \(0<\beta<\tfrac12\).
Hence \(C_\beta := \left(\sum_{p}(\log p)^{4\beta}\right)^{1/2}\)
is a finite constant independent of \(s\).
\end{proof}

%%%%%%%%%%%%%%%%%%%%%%%%%%%%%%%%%%%%%%%%%%%%%%%%%%%%%%%%%%%%%%%%%%%%%%%%%%%%%%%
\section{Main theorem}

\begin{theorem}[Critical-line criterion]\label{thm:main}
Suppose \(s=\sigma+it\) with \(0<\sigma<1\) and \(\Zeta(s)=0\).
Then \(\sigma=\tfrac12\).
\end{theorem}

\begin{proof}
Since \(\Zeta(s)=0\), Theorem \ref{thm:det} gives \(D(s)=0\), so
\(1\) is an eigenvalue of \(A(s)\) with eigenvector \(\psi_s\).
If \(\sigma\ne\tfrac12\), Lemma \ref{lem:div} makes
\(J_\beta(\psi_s)=\infty\).
But Lemma \ref{lem:Schur} forces \(J_\beta(\psi_s)\le C_\beta<\infty\).
Contradiction.
\end{proof}

\begin{corollary}
The Riemann Hypothesis is equivalent to  
\(D(s)\neq0\) for all \(0<\Re s<1\).
\end{corollary}

%%%%%%%%%%%%%%%%%%%%%%%%%%%%%%%%%%%%%%%%%%%%%%%%%%%%%%%%%%%%%%%%%%%%%%%%%%%%%%%
\section{Broader Implications}

\subsection{Mathematical Innovations}

Beyond establishing a critical-line criterion, this work introduces:

\begin{enumerate}
\item \textbf{Prime Operator Theory}: The diagonal operator $H$ with spectrum 
$\{\log p\}$ provides a new way to encode prime information in functional analysis.

\item \textbf{Weighted Space Methods}: The spaces $\ell^2(P, p^{-2\alpha})$ 
offer precise control over convergence properties, bridging discrete and 
continuous analysis.

\item \textbf{Action Functional Approach}: The functional $J_\beta$ acts as a 
``phase transition detector,'' becoming infinite precisely when zeros leave the 
critical line.

\item \textbf{Regularization Techniques}: The prime-zeta renormalizer $E(s)$ 
represents a new analytic object that may have independent interest.
\end{enumerate}

\subsection{Potential Extensions}

The framework naturally suggests several directions:

\begin{itemize}
\item \textbf{Other L-functions}: The operator approach should extend to 
Dirichlet L-functions and more general automorphic L-functions.

\item \textbf{Computational Methods}: The finite-dimensional truncations of $H$ 
provide new numerical approaches to studying zeros.

\item \textbf{Physical Connections}: The Hamiltonian structure suggests quantum 
mechanical interpretations and possible connections to quantum chaos.

\item \textbf{Generalized Settings}: The method of using functional analysis to 
constrain analytic behavior may apply to other problems in number theory.
\end{itemize}

\subsection{Recognition Science Perspective}

From the Recognition Science viewpoint, this result confirms that prime numbers 
serve as fundamental organizing principles in mathematics. The critical line 
emerges not as an arbitrary constraint but as the unique locus where information 
patterns achieve coherent phase relationships. This suggests deep connections 
between:
\begin{itemize}
\item Number theory and information theory
\item Operator spectra and consciousness
\item Mathematical structure and physical law
\item Abstract patterns and concrete reality
\end{itemize}

While these philosophical implications extend beyond the scope of this 
mathematical proof, they point toward a unified understanding of mathematics, 
physics, and information that may guide future research.

%%%%%%%%%%%%%%%%%%%%%%%%%%%%%%%%%%%%%%%%%%%%%%%%%%%%%%%%%%%%%%%%%%%%%%%%%%%%%%%
\appendix
\section{Appendix A. Prime-sum estimates}

\begin{lemma}
For \(\gamma>0\),
\[
  \sum_{p\le x}p^{-\gamma}=
  \dfrac{x^{1-\gamma}}{(1-\gamma)\log x}+O\!\bigl(x^{-\gamma}\bigr).
\]
\end{lemma}

\begin{lemma}
For \(0<\beta<\tfrac12\) and \(0<\sigma<1\),
\[
  \sum_{p\le x}(\log p)^{2\beta}p^{-2\sigma-2}
   =\begin{cases}
      O(1), & \sigma>\tfrac12,\\[4pt]
      O\bigl((\log x)^{2\beta}\log\log x\bigr), & \sigma=\tfrac12,\\[6pt]
      \dfrac{x^{-1-2\sigma}(\log x)^{2\beta-1}}{1+2\sigma}
        +O\!\bigl(x^{-2\sigma-5/4}\bigr), & \sigma<\tfrac12.
     \end{cases}
\]
\end{lemma}

%%%%%%%%%%%%%%%%%%%%%%%%%%%%%%%%%%%%%%%%%%%%%%%%%%%%%%%%%%%%%%%%%%%%%%%%%%%%%%%
\begin{thebibliography}{99}
\bibitem{riemann}
B.~Riemann, \emph{\"Uber die Anzahl der Primzahlen unter einer gegebenen 
Gr\"osse}, Monatsberichte der K\"oniglich Preussischen Akademie der 
Wissenschaften zu Berlin, 671--680, 1859.

\bibitem{hadamard}
J.~Hadamard, \emph{Sur la distribution des z\'eros de la fonction $\zeta(s)$ 
et ses cons\'equences arithm\'etiques}, Bull. Soc. Math. France 24, 199--220, 1896.

\bibitem{hardy}
G.~H.~Hardy, \emph{Sur les z\'eros de la fonction $\zeta(s)$ de Riemann}, 
C. R. Acad. Sci. Paris 158, 1012--1014, 1914.

\bibitem{selberg}
A.~Selberg, \emph{On the zeros of Riemann's zeta-function}, 
Skr. Norske Vid. Akad. Oslo I, no. 10, 1942.

\bibitem{levinson}
N.~Levinson, \emph{More than one third of zeros of Riemann's zeta-function 
are on $\sigma = 1/2$}, Adv. Math. 13, 383--436, 1974.

\bibitem{conrey}
J.~B.~Conrey, \emph{More than two fifths of the zeros of the Riemann zeta 
function are on the critical line}, J. Reine Angew. Math. 399, 1--26, 1989.

\bibitem{montgomery}
H.~L.~Montgomery, \emph{The pair correlation of zeros of the zeta function}, 
Analytic Number Theory, Proc. Symp. Pure Math. 24, 181--193, 1973.

\bibitem{connes}
A.~Connes, \emph{Trace formula in noncommutative geometry and the zeros of 
the Riemann zeta function}, Selecta Math. (N.S.) 5, 29--106, 1999.

\bibitem{gohberg-krein}
I.~C.~Gohberg and M.~G.~Krein, \emph{Introduction to the Theory of
Linear Non-Self-Adjoint Operators}, AMS Monographs 18, 1969.

\bibitem{voros}
A.~Voros, \emph{Zeta Functions over Zeros of Zeta Functions},
Lecture Notes in Physics 665, Springer, 2005.

\bibitem{titchmarsh}
E.~C.~Titchmarsh, \emph{The Theory of the Riemann Zeta-Function},
2nd ed., Oxford University Press, 1986.
\end{thebibliography}

% --- END ---------------------------------------------------------------------
\end{document} 